\chapter{Tareas desempeñadas}
\section{Descripción}
Desde un primer momento, en la empresa se me dió libertad total para elegir en qué ámbito
trabajar. Decidí enfocarme en el back-end, debido a mi preferencia personal.

Durante mi estancia en Okticket, he trabajado en las siguientes características:
\begin{itemize}
	\item Script automático de instalación del entorno local, hecho en Ruby y pensado para
		trabajar utilizando contenedores independientes de Docker.
	\item Migración de versiones a las últimas versiones del framework de la API (Laravel),
		incluyendo la actualización total de todas las dependencias.
	\item Migración y desarrollo de un nuevo sistema de pruebas de carga en Locust (Python),
		con el objetivo de medir las mejoras de rendimiento de las nuevas arquitecturas y el
		beneficio de las actualizaciones a nuevas versiones.
	\item Implementación de nuevas herramientas de code styling automatizadas para PHP{.}
	\item Despliegue de nuevas características en nuevas infraestructuras para la expansión
		de Okticket en México.
	\item Detección y mejoras de rendimiento en partes críticas del código.
	\item Gestión de dashboards en NewRelic.
\end{itemize}

Todas estas labores han sido desarrolladas en contenedores de Docker sobre un sistema
operativo Linux con el objetivo de desarrollar mi habilidad personal de tratar con estos
sistemas y poder desplegar más rápida y fácilmente sobre la arquitectura de Okticket en la
nube de Amazon (AWS).

La mayor parte de las tareas son de tipo “I+D”, es decir,
|TODO| completar

\section{Relación con la titulación}
Pese a que las tecnologías que se utilizan en Okticket tienen poco o nada que ver con lo que
he utilizado en la carrera, la asignatura de Administración de Sistemas me ha ayudado a la hora
de lidiar con algunos problemas durante las prácticas, además de darme a conocer tecnologías
como Docker, aunque fuera por encima.

|TODO| completar
