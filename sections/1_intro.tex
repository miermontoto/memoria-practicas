\chapter{Introducción}
\section{Sobre la empresa}
Okticket es una startup nacida en Gijón en 2017 cuyo producto principal es un servicio software
que escanea automáticamente de tickets y notas de gastos lo que permite reducir los costes y el
tiempo que invierten las empresas en contabilizar y manejar los gastos de viaje de los profesionales.

La empresa tienen su suede principal en el Parque Tecnológico de Gijón, aunque cuenta con un número
de sedes creciente en varios países, como Francia, Portugal o, más recientemente, México. En esta
oficina principal se encuentran los departamentos de ventas y marketing, así como el equipo de
desarrollo y soporte.

Okticket es una de las empresas que más crecen tanto del sector como del propio Parque
Tecnológico. Debido a este rápido crecimiento, el equipo está en constante desarrollo y
cambio, tanto aquí en España como en el resto de sedes. Este crecimiento se refleja
en la recepción de un gran número de galardones y reconocimientos.
\footnote{\href{https://www.linkedin.com/posts/okticket_okticket-en-el-especial-startups-de-forbes-activity-7140622980618903552-UGWK}{Okticket en el especial startups 2023 de Forbes (LinkedIn)}}
\footnote{\href{https://www.elcomercio.es/economia/arcelor-okticket-premios-20230222002438-ntvo.html}{Arcelor y Okticket, premios nacional de Ingeniería Informática (EL COMERCIO)}}
\footnote{\href{https://www.okticket.es/blog/empresa-pyme-innovadora}{Okticket recibe el sello Pyme Innovadora (okticket.es)}}
\footnote{\href{https://www.okticket.es/blog/okticket-empresa-emergente-certificada}{Okticket, empresa emergente certificada (okticket.es)}}

La parte principal del negocio es el núcleo del software como servicio (Software as a
Service en inglés, en adelante \textit{SaaS}), es decir, la aplicación completa tanto
para administradores como para empleados. Este SaaS se oferta a empresas de cualquier
tamaño, cuyo precio final varía en función del número de usuarios, las características
e integraciones que requiera la empresa cliente y el soporte que se ofrezca.

Recientemente se han añadido nuevas propuestas a la cartera de servicios ofertada por
Okticket, como la OKTCard {-} una tarjeta inteligente que gestiona automáticamente los gastos,
así como la inclusión de nuevos ``módulos'' de gestión de gastos y viajes.

\newpage{}
\section{Sobre las prácticas}
En la entrada de cualquier persona a Okticket, los nuevos empleados reciben una serie de
programas de iniciación (dentro de la empresa se conocen como ``onboarding'') por parte de
los directores de cada departamento: marketing, ventas internas y externas, éxito cliente…

Una vez formado, el estudiante en prácticas funciona como un nuevo empleado corriente, y,
por lo tanto, tiene la capacidad de explorar sus áreas de conocimiento preferidas y escoger
en qué campo va a trabajar. Además, gracias a la frecuente colaboración interna que existe
en el equipo de desarrollo, una persona suele además validar y ayudar a otros compañeros
en cualquier tipo de tareas, siempre documentando y registrando tanto el trabajo colaborativo
como el individual.

Durante las prácticas, se trabaja con las tecnologías especificadas en el contrato que
forman parte de la arquitectura: PHP, Vue.js, Laravel, SQL, entre otras. Además, se
trabaja con herramientas de gestión de proyectos como Jira y Confluence, así como con
Git en Bitbucket como herramienta de control y gestión de versiones.

\section{Sobre el departamento}
El departamento de Informática de Okticket (o \textit{zona de desarrollo}) está compuesto por
perfiles de todo tipo: tanto front-end como back-end, expertos y juniors, graduados y autoeducados.
Por encima de todo, es un grupo muy coherente y lleno de buenas personas, dispuestas a ayudar
con cualquier cosa y con una relación de grupo excelente. La comunicación es fluida y
constante, y se fomenta la colaboración entre los miembros del equipo.

Las primeras semanas están dedicadas a comprender cómo están montados los servicios.
Esencialmente, la arquitectura funcional de Okticket se divide en tres grandes partes:
\begin{itemize}
	\item La \textbf{API}, hecha en PHP utilizando el framework Laravel, que gestiona todas las
		peticiones y realiza conexiones a las bases de datos.
	\item El \textbf{gestor}, hecho en Vue.js, que es lo que utilizan los administradores y gestores
		de las empresas cliente.
	\item La \textbf{aplicación móvil}, que es lo que utilizan los empleados en el día a día.
\end{itemize}

\newpage{}
Además de estos tres apartados principales, existen otros campos de especialización
dentro del departamento:
\begin{itemize}
	\item \textbf{Soporte}, que se encarga de gestionar los problemas técnicos que les surgen a los
		clientes que se comunican con atención al cliente.
	\item \textbf{Infraestructura}, que se encarga de mantener la disponibilidad de toda la arquitectura.
	\item \textbf{Implantaciones}, referido a las necesidades puntuales de los clientes a la
		hora de integrar Okticket dentro de sus sistemas de gestión, principalmente los \textit{ERPs}.
\end{itemize}

Pese a la existencia de campos de especialización dentro del equipo, dentro de la zona de
desarrollo no hay una jerarquía establecida, y se fomenta la colaboración entre los miembros
del equipo.

La ``zona de desarrollo'' compone casi un tercio de los empleados de la empresa y se
encuentra localizada un edificio distinto pero cercano a la sede y dentro del propio Parque
Tecnológico.

\section{Sobre el alumno}
Aparte de estar estudiando el grado en Ingeniería Informática en Tecnologías de la Información
en la Escuela Politécnica de Gijón, administro mi propio dominio y una colección de
repositorios de todo tipo en \href{https://github.com/miermontoto}{mi página de GitHub}. Estos
proyectos por cuenta propia me ayudan a complementar el conocimiento adquirido en la carrera a la
hora de trabajar y demostrar lo que valgo.

Mi \href{\target}{página web personal} recoge proyectos personales e incluye un apartado de información
personal.
