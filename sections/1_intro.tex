\chapter{Introducción}
\section{Sobre la empresa}
Okticket es una startup nacida en Gijón en 2017 cuyo producto principal es reducir los costes
y el tiempo invertido en las empresas en contabilizar y manejar los gastos de viaje de los
profesionales mediante software de gestión de tickets.

Sus oficinas principales (incluyendo la zona de desarrollo) se encuentran en el Parque
Tecnológico de Gijón, aunque cuenta con un número de sedes creciente en varios países:
Francia, Portugal y, más recientemente, México.

Okticket es una de las empresas que más crecen tanto del sector como del propio Parque
Tecnológico. Debido a este rápido crecimiento, el equipo está en constante desarrollo y
cambio, tanto aquí en España como en el resto de sedes.

Pese a que la parte principal del negocio es el SaaS (Software as a Service en inglés),
es decir, la aplicación completa tanto para administradores como para empleados,
recientemente se han añadido nuevas propuestas como la OKTCard, una tarjeta inteligente
que gestiona automáticamente los gastos, entre otros proyectos.

\section{Sobre la empresa}
En la entrada de cualquier persona a Okticket, los nuevos empleados reciben una serie de
programas de iniciación (dentro de la empresa se conocen como onboarding) por parte de
los directores de cada departamento: marketing, ventas internas y externas, éxito cliente…

Una vez formado, el estudiante en prácticas funciona como un nuevo empleado corriente, y
por lo tanto tiene la capacidad de explorar sus áreas de conocimiento preferidas y escoger
en qué campo va a trabajar. Además, gracias a la frecuente colaboración interna que existe
en el equipo de desarrollo, una persona suele trabajar además validando o ayudando a otros
empleados en cualquier tipo de tareas, siempre documentando y registrando tanto el trabajo
colaborativo como el individual.

Durante las prácticas, se trabaja con las tecnologías especificadas en el contrato que
forman parte de la arquitectura: PHP, Vuejs, COMPLETAR

|TODO| completar

\section{Sobre el departamento}
El departamento de Informática de Okticket (o Zona de Desarrollo) está compuesto por perfiles
de todo tipo: tanto front-end como back-end, expertos y juniors, graduados y autoeducados.
Por encima de todo, es un grupo muy coherente y lleno de buenas personas, dispuestas a ayudar
con cualquier cosa y con una relación de grupo excelente.

Las primeras semanas están dedicadas a comprender cómo están montados los servicios.
Esencialmente, la arquitectura funcional de Okticket se divide en tres grandes partes:
\begin{itemize}
	\item La API, hecha en PHP utilizando el framework Laravel, que gestiona todas las
		peticiones y realiza conexiones a las bases de datos.
	\item El gestor, hecho en Vue.js, que es lo que utilizan los administradores y gestores
		de las empresas cliente.
	\item La aplicación móvil, que es lo que utilizan los empleados en el día a día.
\end{itemize}

Aunque existan estos tres apartados principales, existen otros campos de especialización
dentro del departamento:
\begin{itemize}
	\item Soporte, que se encarga de gestionar los problemas técnicos que les surgen a los
		clientes que se comunican con atención al cliente.
	\item Infraestructura, que se encarga de mantener la disponibilidad de toda la arquitectura.
	\item Implantaciones, referido a las necesidades puntuales de los clientes a la
		hora de integrar Okticket dentro de sus sistemas de gestión, principalmente los \textit{ERPs}.
\end{itemize}

Aunque estos son los principales campos de desarrollo, no hay títulos ni roles dentro del
equipo y cualquier desarrollador puede completar tareas de diferentes campos.

La “zona de desarrollo” compone casi un tercio de los empleados de la empresa, por lo que se
encuentra localizada un edificio distinto pero cercano a la sede y dentro del propio Parque
Tecnológico.

\section{Sobre el alumno}
Aparte de estar estudiando el grado en Ingeniería Informática en Tecnologías de la Información
en la Escuela Politécnica de Gijón, administro mi propio dominio y una colección de
repositorios de todo tipo en mi página de GitHub. Estos proyectos por cuenta propia me ayudan
a complementar el conocimiento adquirido en la carrera a la hora de trabajar y demostrar lo que
valgo.

Mi \href{\target}{página web personal} recoge mis proyectos personales e incluye un apartado de información
personal.
